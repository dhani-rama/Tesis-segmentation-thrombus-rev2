\begin{spacing}{1}
\begin{center}
		\large\textbf{\JdTesis}
	\end{center}
	\normalsize
	\begin{adjustwidth}{-0.2cm}{}
		\ifthenelse{\boolean{bMaster}}{
		
		\begin{tabular}{lcp{0.9\linewidth}}
		Nama Mahasiswa &:& \NamaMahasiswa\\
			NRP &:&\NrpMahasiswa\\
			Pembimbing &:& 1. \PbSatu\\
			\ifthenelse{\boolean{PembimbingDua}}{& & 2. \PbDua\\}{}
			
			\ifthenelse{\boolean{PembimbingTiga}}{& & 3. \PbTiga\\}{}
			
			
			
		\end{tabular}
	}{
	\begin{tabular}{lcp{0.7\linewidth}}
		Nama Mahasiswa &:& \NamaMahasiswa\\
		NRP &:&\NrpMahasiswa\\
		Promotor &:&  \PbSatu\\
		\ifthenelse{\boolean{PembimbingTiga}}{
		\ifthenelse{\boolean{PembimbingDua}}{Co. Promotor&: & 1. \PbDua\\}{}
		\ifthenelse{\boolean{PembimbingTiga}}{& & 2. \PbTiga\\}{}
	}
{
	\ifthenelse{\boolean{PembimbingDua}}{\hspace{5ex}Co. Promotor&: & \PbDua\\}{}
	
}
	\end{tabular}

}

	
	\end{adjustwidth}
	\vspace{2ex}
	\begin{center}
		\Large\textbf{ABSTRAK}
	\end{center}
	\vspace{1ex}	
%Tulis Abstrak disini
\textit{Deep Vein Thrombosis} (DVT) merupakan sebuah penyakit yang diakibatkan adanya pembentukan thrombus pada pembuluh darah vena dalam. Thrombus ini dapat mengganggu aliran darah normal dan menyebabkan masalah serius apabila tidak diobati. Dataset yang digunakan dalam penelitian ini berupa citra 2D \textit{ultrasound thrombus} 5 pasien penderita DVT dan citra \textit{thrombus} dan pembuluh darah \textit{phantom} balon panjang. Diagnosis thrombus apabila dilakukan secara manual memerlukan waktu yang tidak sebentar serta analisis akurasi pembacaan citra \textit{thrombus} bergantung pada dokter spesialis. Oleh karena itu, diperlukan adanya diagnosis \textit{thrombus} untuk penderita DVT secara otomatis guna mempersingkat waktu serta meningkatkan performa analisis akurasi pembacaan citra \textit{thrombus}. Penelitian ini mengusulkan segmentasi 2D dan 3D \textit{thrombus} pada citra \textit{ultrasound} \textit{phantom} balon panjang menggunakan model segmentasi U-Net. Penelitian ini berhasil melakukan segmentasi gumpalan darah vena pada citra \textit{ultrasound} menggunakan U-Net 3D. Berdasarkan hasil segmentasimodel segmentasi U-Net 3D mendapat nilai \textit{accuracy} sebesar 99,1078\% dan nilai \textit{loss} sebesar 0,0208. Berdasarkan perhitungan evaluasi metrik untuk perhitungan antara hasil citra prediksi dan groundtruth dengan menggunakan IoU, \textit{dice coefficient}, dan \textit{hausdorff distance}, citra 3D \textit{ultrasound} \textit{thrombus} dan pembuluh darah mendapat nilai \textit{mean} IoU sebesar 0,8105, \textit{mean dice coefficient} sebesar 0,8953, dan \textit{mean hausdorff distance} sebesar 3,25. Pada segmentasi 2D citra \textit{ultrasound} \textit{thrombus} dan pembuluh darah \textit{phantom} balon panjang, penggunaan \textit{encoder} \textit{pre-trained VGG16} pada model U-Net 2D dapat meningkatkan kinerja model untuk segmentasi area \textit{thrombus}. Penerapan peningkatan kualitas citra dengan filter \textit{gaussian} dan filter \textit{median} memberikan pengaruh dalam peningkatan performa segmentasi 2D. \\

%Tulis Kata Kunci disini
\vspace{2ex}
\textbf{Kata kunci }: \textit{Deep Vein Thrombosis},  Citra \textit{Ultrasound}, segmentasi, \textit{pre-trained} VGG16 and UNet, filter \textit{denoising}
	
\end{spacing}