\begin{center}
\Large\textbf{Kata Pengantar}
\end{center}
\vspace{2ex}
%Tulis kata pengantar di sini
Puji dan syukur kehadirat Allah SWT atas segala limpahan berkah, rahmat, serta hidayah-Nya, penulis  dapat menyelesaikan laporan ini dengan judul SEGMENTASI GUMPALAN DARAH VENA PADA CITRA ULTRASOUND MENGGUNAKAN U-NET.

Pada penulisan buku tesis ini, penulis memiliki banyak kekurangan dan keterbatasan sehingga proses penyusunan buku tesis ini tidak lepas dari bantuan, dukungan, dan bimbingan dari berbagai pihak. Oleh karena itu, dengan hormat, penulis mengucapkan terima kasih kepada:

\begin{enumerate}
	\item Prof. Dr. I Ketut Eddy Purnama, S.T., M.T. dan Dr. Eko Mulyanto Yuniarno, S.T., M.T. yang telah mengarahkan, memberikan semangat, saran, dan masukan dalam penyusunan buku tesis.
	\item Dr. Eko Mulyanto Yuniarno, S.T., M.T. selaku koordinator bidang keahlian Jaringan Cerdas Multimedia.
	\item Dewan penguji yang telah memberikan masukan dan arahan dalam tesis ini.
	\item Orang tua dan keluarga yang selalu memberikan bantuan dalam bentuk moral, material, dan doa kepada penulis.
	\item Teman - teman seperjuangan JCM yang telah memberikan dukungan dalam pengerjaan tesis ini.
	\item Teman - teman Lab Visi Komputer (B300) yang memberi semangat, kritik, dan saran, serta dukungan dalam penyelesaian buku tesis ini.
	\item Semua pihak yang telah membantu dalam proses pengerjaan buku tesis ini.
\end{enumerate}

 \vspace{1ex}
Dalam penyusunan buku tesis ini, penulis menyadari bahwa masih terdapat kekurangan dalam segi penulisan maupun isi dari tesis. Oleh karena itu, kritik dan saran yang membangun sangat diharapkan oleh penulis
	\vspace{26pt}
	\begin{flushright}
		\begin{tabular}[b]{c}
			Surabaya, 21 Desember 2023
			\\
			\\
			\\
			Penulis
		\end{tabular}
	\end{flushright}
