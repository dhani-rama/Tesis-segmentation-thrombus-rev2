\begin{spacing}{1}
\begin{center}
		\large\textbf{\JdTesisEng}
	\end{center}
	\normalsize
	\begin{adjustwidth}{-0.2cm}{}
		\begin{tabular}{lcp{0.9\linewidth}}
		By &:& \NamaMahasiswa\\
			Student Identity Number &:&\NrpMahasiswa\\
			Supervisor &:& 1. \PbSatu\\
			\ifthenelse{\boolean{PembimbingDua}}{& & 2. \PbDua\\}{}
			\ifthenelse{\boolean{PembimbingTiga}}{& & 3. \PbTiga\\}{}
		\end{tabular}
	\end{adjustwidth}
	\vspace{2ex}
	
	\begin{center}
		\Large\textbf{ABSTRACT}
	\end{center}
	\vspace{1ex}	
%Tulis Abstrak disini
Deep Vein Thrombosis (DVT) is a disease that occurs when a thrombus forms within the deep veins. This thrombus can disrupt normal blood flow and lead to severe issues if left untreated. The dataset used in this research consists of 2D ultrasound images of thrombus from 5 patients with DVT. Manual thrombus diagnosis requires a considerable amount of time, and the accuracy of thrombus image analysis relies on specialized doctors. Hence, an automatic thrombus diagnosis is needed for DVT patients to shorten the time and enhance the accuracy of thrombus image analysis. This research proposes thrombus segmentation in ultrasound images using pre-trained VGG16 and UNet model based on denoising filters. The encoder for the UNet model in this segmentation model is a pre-trained VGG16 model. In this study, five denoising filters are utilized.Based on the conducted experiments, the Gaussian filter yielded the most optimal results for thrombus segmentation with an accuracy of 99.166\% and a loss value of 0.0269 for the UNet model. Furthermore, the pre-trained VGG16 and UNet model's accuracy was 99.222\%, and the loss value was 0.284. Thrombus prediction tests using the UNet model resulted in a mean IoU of 77.087\%, a mean Dice coefficient of 0.8608, and a mean Hausdorff distance of 3.44. Meanwhile, thrombus prediction tests using the pre-trained VGG16 and UNet model produced a mean IoU of 88.298\%, a mean Dice coefficient of 0.8784, and a mean Hausdorff distance of 3.07. As a result, utilizing VGG16 as the encoder in the UNet architecture may enhance accuracy when segmenting. \\

%Tulis Kata Kunci disini
\vspace{2ex}
\textbf{Keyword}: Deep Vein Thrombosis,  Ultrasound Image, Segmentation, Pre-trained VGG16 and UNet, Denoising Filter
\end{spacing}